
%--------------------------------------------------------------------
%--------------------------------------------------------------------
% Formato para los talleres del curso de Métodos Computacionales
% Universidad de los Andes
%--------------------------------------------------------------------
%--------------------------------------------------------------------

\documentclass[11pt,letterpaper]{exam}
\usepackage{amsmath}
\usepackage[utf8]{inputenc}
\usepackage[spanish]{babel}
\usepackage{graphicx}
\usepackage{tabularx}
\usepackage[absolute]{textpos} % Para poner una imagen completa en la portada
\usepackage{hyperref}
\usepackage{float}

\newcommand{\base}[1]{\underline{\hspace{#1}}}
\boxedpoints
\pointname{ pt}

\extraheadheight{-0.15in}

\newcommand\upquote[1]{\textquotesingle#1\textquotesingle} % To fix straight quotes in verbatim



\begin{document}
\begin{center}
{\Large Universidad de los Andes - M\'etodos Computacionales Avanzados} \\
Tarea 1 - \textsc{MPI}\\
24-02-2017\\
\end{center}

\vspace{0.3cm}


\noindent
La soluci\'on a este ejercicio debe subirse por SICUA antes de las 6:00PM
del viernes 10 de marzo del 2017. 
Los c\'odigos deben encontrarse en un unico repositorio de \verb'github'
con el nombre \verb"NombreApellido_Tarea1". Por ejemplo yo deber\'ia
subir crear un repositorio con el nombre \verb"JaimeForero_Tarea1". 

\noindent
En el repositorio deben estar los siguientes elementos.
\begin{itemize}
\item (60 puntos) Un c\'odigo fuente en C paralelizado en OPENMP que resuelve las ecuaciones diferenciales y produce datos.
\item (20 puntos) Un c\'odigo en Python que lee los datos producidos por el
  c\'odigo en C y produce visualizaciones.
\item (20 puntos) Un c\'odigo en Python que grafica el tiempo en que corre el c\'odigo como funci\'on del num\'ero de procesadores.
\item (10 puntos) Un makefile que: 1) compila el c\'odigo en C 2) ejecuta el programa con 1, 2 y 4 procesadores y 3) produce graficas en Python.
\item (10 puntos) Un archivo de texto (\verb"README") con nombres completos y c\'odigos de los integrantes (m\'aximo dos personas).
\end{itemize}

\vspace{0.3cm}

\begin{questions}
\question{\bf{Problema Fermi-Pasta-Ulam-Tsingou}}

Considere un s\'olido unidimensional con N \'atomos.
Vamos a pensar que los \'atomos est\'an unidos por acoplamientos no
lineales de resortes, de tal manera que los desplazamientos con respecto
a su posici\'on de equilibrio est\'an descritos por

\begin{equation}
\ddot{x}_n = (x_{n+1} - 2x_n + x_{n-1}) + \beta[(x_{n+1}-x_n)^3 - (x_n - x_{n-1})^3]
\end{equation}

Tomando condiciones de contorno fijas $x_{0}=x_{N-1}=0$, modele el comportamiento
de este s\'olido unidimensional un tiempo total de evoluci\'on $T=5.0N^{2.2}$
usando $\beta=1.0$ y $N=64$. La condici\'on inicial debe ser
$x_n = \sin(\pi n/(N-1))$. La integraci\'on
temporal debe hacerse con un m\'etodo leapfrog y $\Delta t=5\times 10^{-3}$.

El c\'odigo debe producir como resultado un archivo de la energ\'ia en los modos $k=1,2,3$ de toda la cadena para $1000$ instantes equiespaciados
temporalmente (ver las referencias para tener la expresi\'on
expl\'icita de $E_k$).

Luego un archivo de python debe leer este archivo de salida para
producir una gr\'afica de $E_k$ en funci\'on del tiempo para estos
primeros tres modos.  

Referencias sobre este problema\\
\url{http://www.scholarpedia.org/article/Fermi-Pasta-Ulam_nonlinear_lattice_oscillations}\\
\url{http://www.me.umn.edu/~dtraian/FPU.pdf}\\
\end{questions}
\end{document}

