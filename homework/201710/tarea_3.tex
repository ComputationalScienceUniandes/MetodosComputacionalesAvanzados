
%--------------------------------------------------------------------
%--------------------------------------------------------------------
% Formato para los talleres del curso de Métodos Computacionales
% Universidad de los Andes
%--------------------------------------------------------------------
%--------------------------------------------------------------------

\documentclass[11pt,letterpaper]{exam}
\usepackage{amsmath}
\usepackage[utf8]{inputenc}
\usepackage[spanish]{babel}
\usepackage{graphicx}
\usepackage{tabularx}
\usepackage[absolute]{textpos} % Para poner una imagen completa en la portada
\usepackage{hyperref}
\usepackage{float}

\newcommand{\base}[1]{\underline{\hspace{#1}}}
\boxedpoints
\pointname{ pt}

\extraheadheight{-0.15in}

\newcommand\upquote[1]{\textquotesingle#1\textquotesingle} % To fix straight quotes in verbatim



\begin{document}
\begin{center}
{\Large Universidad de los Andes - M\'etodos Computacionales Avanzados} \\
Tarea 3 - \textsc{Machine Learning}\\
05-05-2017\\
\end{center}

\vspace{0.3cm}


\noindent
La soluci\'on a este ejercicio debe subirse por SICUA antes de las 6:00PM
del lunes 22 de Mayo del 2017. 
Los c\'odigos deben encontrarse en un unico repositorio de \verb'github'
con el nombre \verb"NombreApellido_Tarea3". Por ejemplo yo deber\'ia
subir crear un repositorio con el nombre \verb"JaimeForero_Tarea3". 

\noindent
En el repositorio deben estar un notebook de jupyter (Python3,
Scikit-Learn 0.18) que responde a las preguntas planteadas.
\vspace{0.3cm}

\begin{questions}
\question{\bf{Haciendo Planetas}}

Los planetas se forman a partir de un disco de gas y polvo que rodeaa
la estrella central del sistema. A partir de un c\'odigo que sigue la
evoluci\'on de este disco protoplanetario se pueden producir miles de
sistemas planetarios diferentes variando los par\'ametros de
entrada. Este sistema es altamente no lineal y no es claro cu\'ales
variables de entrada dominan en los sistemas planetarios producidos.
Uds. van a intentar encontrar patrones en estos resultados usando
Machine Learning.

En este lugar

\url{https://github.com/ComputoCienciasUniandes/MetodosComputacionalesDatos/tree/master/homework/2017-10/planetas}

se encuentran resultados de estas simulaciones. La estructura de los
archivos se encuentran descrita en el README.

Para cada uno de estos archivos, utilizando m\'etodos de machine
learning construya modelos que puedan predecir:

\begin{itemize}
\item (20 puntos) N\'umero de planetas
\item (20 puntos) Masa total de planetas (sin contar masa de gas)
\item (20 puntos) Eficiencia de masa (masa total de planetas/masa del disco)
\end{itemize}

Plantee otros modelos de machine learning para encontrar los
  par\'ametros de entrada que m\'as influyen en la predicci\'on de 
\begin{itemize}
\item (20 puntos) N\'umero de planetas
\item (20 puntos) Masa total de planetas (sin contar masa de gas)
\item (20 puntos) Eficiencia de masa (masa total de planetas/masa del
  disco)
\end{itemize}
\end{questions}
\end{document}

